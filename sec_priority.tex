\section{Research Community Priority}\label{sec:priority}

The need for a next-generation liquid xenon TPC is strongly acknowledged throughout the international particle physics community. 

Studies towards a large-scale liquid xenon dark matter detector started already in 2009 within the EU-ASPERA program which eventually led to the DARWIN project. The support for DARWIN was strongly recommended in the 2011 update of the ASPERA roadmap~\cite{ASPERA2011}. During the ``Snowmass" process to plan research priorities in 2013, U.S. particle physicists concluded that the discovery goal of liquid xenon dark matter detectors must be to ``search for WIMPs over a wide mass range (1~GeV to 100~TeV)... until we encounter the coherent neutrino scattering signal that will arise from solar, atmospheric and supernova neutrinos."~\cite{Snowmass:2013}. In 2017, the Astro Particle Physics European Consortium (APPEC) devised a European Strategy, which aimed to converge ``with its global partners" on the realization of at least one ``ultimate Dark Matter detector based on xenon"~\cite{APPEC:2017}. 

The Division of Particles and Fields of the American Physical Society defined the next step for the detection of WIMPs to be "to partner with Europe and Asia on one large international generation-3 detector"~\cite{DPF:2018a} and they note that detector R\&D looks promising for ``the scaling up of liquid noble...detectors to cover the WIMP mass range to the coherent neutrino floor"~\cite{DPF:2018b}. The Chinese community also endorses a next generation deep underground xenon observatory as one of the top priorities in particle astrophysics~\cite{China:2021}, and the APPEC Dark Matter Report states that underground dark matter programmes with the sensitivity to reach down to the ``neutrino floor at the shortest possible timescale'' should receive enhanced support~\cite{Billard:2021uyg}. Clearly, the main goal is to search for dark matter, but it is understood that such ultimate detector will have other important implications for astrophysics and the quest for the nature of neutrinos. This whitepaper is a response to the global support for a next-generation liquid xenon TPC, as evidenced here.

\subsection{Dark Matter}

In the past two decades, the goal of liquid xenon TPCs has been to detect theorized elastic scatters of WIMP dark matter off xenon nuclei. In addition to WIMPs, these detectors have sensitivity to a large host of well-motivated dark matter candidates, as outlined in this work. About 10~different xenon-based dark matter detectors were built over the years, increasing the xenon target mass by almost three orders of magnitude, reducing the electronic recoil background by about four orders of magnitude and improving the sensitivity to WIMP dark matter by more than a factor~1000. After the pioneering work by ZEPLIN-II/III and XENON10, XENON100, LUX and PandaX managed to build a suite of detectors with world-leading sensitivity. XENON1T was the first TPC with a target above the ton-scale. The current generation of detectors, XENONnT~\cite{Aprile:2020vtw}, LUX-ZEPLIN (LZ)~\cite{Akerib:2018lyp}, and PandaX-4T~\cite{Zhang:2018scp}, feature multi-ton liquid xenon targets. Despite a lack of definitive signal so far, these detectors are clear leaders in sensitivity to WIMPs and other physics channels, and scale reliably in mass~\cite{Aprile:2018dbl}. It is for these reasons that a next-generation liquid xenon TPC is of such interest to the international dark matter community.  

The Update of the European Strategy for Particle Physics (ESPP) from 2020 points out that the search for dark matter is a crucial part of the search for new physics and that experiments that offer ``potential high-impact" should be supported~\cite{ESPP:2020}. The APPEC Report on the Direct Detection of Dark Matter (2021) states that ``the search for dark matter with the aim of detecting a direct signal of dark matter particle interactions with a detector should be given top priority in astroparticle physics, and in all particle physics"~\cite{Billard:2021uyg}. Already in 2014, the U.S.~Particle Physics Project Prioritization Panel (P5) highlighted the identification of the new physics of dark matter as one of the five science drivers for all of particle physics and recommended that U.S.~funding agencies ``support one or more third-generation (G3) direct detection experiments...[with] a globally complementary program and increased international partnership in G3 experiments"~\cite{P5:2014}. A first consolidation of the world-wide xenon community took place when the members of the ZEPLIN collaboration joined LUX, and XMASS teamed up with XENON. Another important political step towards the realization of the next-generation detector happened in 2021, when the scientists from XENON/DARWIN and LUX-ZEPLIN agreed to join forces. 

The German~\cite{GerPS:2018}, Swiss~\cite{chipp2021} and Dutch~\cite{NL:2014} particle physics communities likewise identified the multi-ton liquid xenon observatory DARWIN of particular interest for their national strategy roadmaps and support R\&D towards this goal via national funding programs. Other countries are strong members of the XENON experiment and it is expected that its follow-up project (e.g., DARWIN) will also be supported. The U.K.’s Particle Astrophysics roadmap also stresses the importance of a Xe-based G3 observatory, explicitly recommending R\&D towards G3 as the highest priority in dark matter. R\&D was also supported by European Research Council (ERC) grants.


\subsection{Neutrinoless Double Beta Decay}

Understanding the physics of neutrino mass is another important science driver for particle physics identified by the U.S. P5 and APPEC, which noted the importance of neutrinoless double beta decay searches in that context~\cite{P5:2014,APPEC:2017}. Such experiments are also a top priority in the 2015 US Long Range Plan for Nuclear Science~\cite{LRP:2015}, with one of the four main recommendations being the construction of a massive detector. The European APPEC double beta report (2019) states that "the search for neutrinoless double beta decay is a top priority in particle and astroparticle physics" and acknowledges that the $0\nu\beta\beta$ sensitivity of a next generation \emph{dark matter} detector opens up an exciting scenario~\cite{Giuliani:2019uno}. Similar statements of support for neutrinoless double beta decay detection, especially in dark matter detectors, can be found in UK~\cite{UK:2015}, Russian~\cite{Russia:2012}, CERN/European~\cite{CERN:2013}, and Chinese~\cite{China:2020} particle and nuclear physics priority planning documents. The APPEC Dark Matter Report states on this topic that ``the potential of dark matter detectors to search for rare nuclear decays has been demonstrated spectacularly when XENON1T observed for the first time double electron capture on $^{124}$Xe~\cite{XENON:2019dti}"~\cite{Billard:2021uyg}. 

\subsection{Neutrinos}

Recently, the observed phenomenon of coherent elastic neutrino-nucleus scattering~\cite{Akimov:2017ade} has made large dark matter detectors, such as the one discussed here, particularly desirable for studying neutrinos. Such a detector would be invaluable to the field of astrophysics for measuring Galactic supernovae neutrinos of all flavors. A next-generation liquid xenon detector would be able to probe multiple solar, atmospheric and supernova neutrino signals, which are invaluable measurements in their own right. The US Nuclear Physics community, in the 2015 Long Range Plan for Nuclear Science (LRP)~\cite{LRP:2015}, notes that measuring the CNO cycle and addressing the ``metallicity problem" in the Sun --- both accessible to this detector technology --- are the next big challenges in solar neutrino research.

Taken together, the experiment discussed here addresses a number of high-priority science issues. Spanning across (astro-)particle physics, astrophysics, and nuclear physics, such a detector will significantly advance fundamental science at a variety of fronts.
