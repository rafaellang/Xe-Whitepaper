\section{Summary}\label{sec:summary}

The compelling and versatile science case for a next-generation liquid xenon experiment, combined with its mature technology and minimal technological risk, renders such a detector a paramount facility for the next decade of particle physics, nuclear physics, and astrophysics. This detector will be sensitive to many types of dark matter interactions. Probing the remaining, well-motivated parameter space for spin-independent WIMP scattering down to the neutrino fog will be a milestone in the quest to unravel the nature of dark matter. With its xenon target, this detector will have unprecedented sensitivity to a variety of dark matter models, including spin-dependent couplings, axion-like particles, dark photons, and sterile neutrinos. With the help of optimized analyses, it covers dark matter masses ranging from kilo-electronvolts all the way up to the Planck mass. This next-generation experiment will therefore have significant and lasting impact on dark matter physics.

Simultaneously, such a next-generation liquid xenon experiment will be a competitive experiment in the search for neutrinoless double-beta decay, using a very cost-effective natural xenon target. It will therefore directly address one of the most pressing problems of nuclear physics. Isotopic separation of the natural xenon target can be used to further this sensitivity, or to enable a direct measurement of solar CNO neutrinos. 

Furthermore, this next-generation experiment will be a true observatory for a number of relevant physics. Examples include a precision measurement of the Solar pp neutrino flux, a measurement of the Solar metallicity through boron-8 neutrinos, as well as a first measurement of atmospheric neutrinos in the mega-electronvolt energy range. This detector also has the chance to observe neutrinos from a Galactic supernova in a complementary, flavor-independent channel, if such an event were to occur in the lifetime of the experiment. 

Finally, this detector provides the opportunity to search for a host of signatures from physics beyond the standard model of particle physics. No other technology is capable of probing this many different signals, spanning areas from cosmology to nuclear physics, particle physics, and solar astrophysics.

